% !Tex root=main.tex

\section{Theory of strings and the straight-line fragment}
\label{sec:preliminaries}

This section introduces the theory of strings and gives an example on how such string constraints can look like and how they relate to practical applications.
This paper follows the definition of the theory of strings as introduced in ~\cite{??}


%---------- Basic Notations ----------
\paragraph*{\bfseries Basic Notations.}
Let $\Sigma$ be a finite alphabet. $\Sigma^*$ is the set of finite words over $\Sigma$, and $\epsilon$ is the empty word. The concatenation of strings $u$, $v$ is denoted by $ v \circ  w$, written short as $vw$. For a word $w \in \Sigma^*$, $| w |$ denotes the length of $w$. 
For any word $w = a_1 ... a_n$, we denote by $w[i]$ the letter $a_i$.

\paragraph{Regular languages.}A regular language over a finite alphabet $\Sigma$ is a subset of $\Sigma*$ and can be built as follows. 
The empty language $\emptyset$ is a regular language.
For each $a \in Sigma$, $\{a\}$ is a regular language.
For two regular languages $A,B$, application of the operations of concatenation, union and Kleene star yields a regular language.
\paragraph{Rational relations.} An $n$-ary relation $R$ over a finite alphabet $\Sigma$ is a subset of $(\Sigma^*)^n$ which can be seen as an $n$-tuple over the alphabet of a regular language. A word $(w_1,\dots,w_n)$ is in $R$ if and only if $(a_1^1,\dots,a_n^1),\dots, (a_1^k,\dots,a_n^k) \in L$, and $w_i = a_1 \circ \dots \circ a_k$ with $k$ denoting the length of the word $w_i$.

\paragraph{String language.} String constraints are defined over a finite alphabet $\Sigma$ and a finite set of variables over $\Sigma*$.
The syntax of a formula $\phi$ is given in Figure \ref{fig:syntax-string}.

\begin{figure}
\begin{align*}
\phi &\Coloneqq x = t \mid P(x) \mid \mathcal{R}(\bar{x}) \mid \phi \wedge \phi \mid \phi \vee \phi \mid \lnot \phi \\
t &\Coloneqq x \mid a \mid t \circ t
\end{align*}
\caption{Syntax of string formulas}
\label{fig:syntax-string}
\end{figure}
The formula $\phi$ is a Boolean combinations of formulas of \emph{word equations} $x = t$, \emph{regular constraints} $P(x)$ and \emph{rational constraints} $\mathcal{R}(\bar{x})$ where $x$ is a string variable, $\bar{x}$ is a vector of string variables, and $a \in \Sigma$.
$P(x)$ is a regular language membership query and $\mathcal{R}(\bar{x})$ is a rational relation membership query with $P \subseteq \Sigma^*$ and $\mathcal{R}\subseteq (\Sigma^*)^n$.
The regular languages will be represented by alternating finite automata and the rational constraints by alternating finite transducers.

A formula $\phi$ is satisfiable if and only if there is an \emph{assignment} $\nu : var(\phi) \mapsto \Sigma^*$ such that $\nu$ satisfies $\phi$, otherwise the formula is unsatisfiable.

\begin{example}\label{example:pre}
Consider following Javascript snippet from Hol\'{i}k et al. ~\cite{??}
\begin{lstlisting}*[caption=My Javascript Example]
var x = goog.string.htmlEscape(name);
var y = goog.string.escapeString(x);
nameElem.innerHTML = '<button onclick= "viewPerson(\'' + y + '\')">' + x + '</button>';
\end{lstlisting}

The code assigns an HTML markup for a button to the element \texttt{nameElem}.
When clicking on the button, the function \texttt{viewPerson} gets called with the input \texttt{y}. The variable \texttt{y} is a sanitization of the input \texttt{name} which is done by using \texttt{htmlEscape} and \texttt{escapeString}.
The code can be expressed as a string formula as follows:
\begin{align*}
x = \mathcal{R}_1(name) \wedge y = \mathcal{R}_2(x) \wedge z = w_1 \circ y \circ w_2 \circ x \circ w_3 \wedge u = \mathcal{R}_3(z)
\end{align*}

Note that a rational constraint $\mathcal{R}(x,y)$ can also be written as $x = \mathcal{R}(y)$. 
Furthermore, $name,\;x,\; y,\; z$ and $u$ are variables of the string formula. 
$\mathcal{R}_1$ is a rational constraints which corresponds to the \texttt{htmlEscape} method, $\mathcal{R}_2$ corresponds to the \texttt{escapeString} method, and $\mathcal{R}_3$ corresponds to an implicit transduction that HTML does when calling \texttt{innerHTML}.
The concatenation consists of the constants $w_1, \; w_2$ and $w_3$ which encode the constant strings used in line 3 and are concatenated with $y$ and $x$.
For checking if this string formula accepts any malicious inputs, another regular constraint $\mathcal{P}(z)$ would need to be concatenated which contains all malicious inputs.
If the formula is satisfiable we found a malicious input, if it is unsatisfiable we proofed that the our program accepts no malicious inputs.
Thus, for the rest of the paper we are interested in showing that formulas are unsatisfiable.

\end{example} 

\paragraph*{Straight-line fragment.} Informally you can describe formulas in the straight-line fragment as formulas that disallow branches and loops. 
Formally, a straight-line conjunction is a string constraint which can be written as $\phi \wedge \bigwedge^m_{i=1} x_i = P_i$.
The formula $\phi$ is a conjunction of regular and negated regular constraints.
The $P_i$ are either rational constraints of the form $\mathcal{R}(y)$ or concatenations $y_1 \circ \dots \circ y_n$ and  $P_i$ cannot contain the variables $x_i,\dots x_m$.
A formula is in the straight-line fragment if every clause in its disjunctive normal form is a straight-line conjunction.

Example \ref{example:pre} is a formula in the straight-line fragment.
Note that the rational constraints here are binary and do not allow $n$-ary relations.
The first step that \sloth does is translate a straight-line formula into a formula that is in the acyclic fragment.
This report will not go into further detail on how this transformations works, however, this is the first source of exponential blow up which cannot be avoided as the straight-line fragment is EXSPACE-complete and the acyclic fragment is PSPACE-complete.

